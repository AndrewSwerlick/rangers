\documentclass[11pt,letterpaper,openright]{book}

% Essential packages
\usepackage[utf8]{inputenc}
\usepackage[T1]{fontenc}
\usepackage{graphicx}
\usepackage{geometry}
\usepackage{hyperref}
\usepackage{parskip}

% Page geometry for print
\geometry{
  paperwidth=6in,
  paperheight=9in,
  margin=0.75in,
  top=1in,
  bottom=1in
}

% PDF metadata
\hypersetup{
  pdftitle={Your TTRPG Rulebook},
  pdfauthor={Your Name},
  pdfsubject={Tabletop RPG Rules},
  pdfkeywords={TTRPG, RPG, Game, Rules},
  colorlinks=true,
  linkcolor=black,
  urlcolor=blue
}

% Title information
\title{Your TTRPG Rulebook}
\author{Your Name}
\date{\today}

\begin{document}

% Front matter
\frontmatter

\maketitle

\tableofcontents

\chapter{Preface}

Welcome to your TTRPG! This is a template to get you started.

% Main content
\mainmatter

\chapter{Introduction}

\section{What is this game?}

Your game description goes here.

\section{What you need to play}

\begin{itemize}
  \item Dice (specify which kinds)
  \item Character sheets
  \item Pencils and paper
  \item Imagination!
\end{itemize}

\chapter{Character Creation}

\section{Choosing your character}

Details about character creation...

\section{Attributes}

Define your game's attributes here.

\chapter{Core Rules}

\section{Basic mechanics}

Explain your core game mechanics.

\section{Combat}

How combat works in your game.

% Back matter
\backmatter

\chapter{Appendices}

\section{Quick Reference Tables}

Add useful tables for quick reference during gameplay.

\end{document}
